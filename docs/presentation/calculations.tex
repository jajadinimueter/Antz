\subsection{Berechnungen}

\frame{
	\frametitle{Berechnungen}
	\begin{description}
		\item [Kantenauswahl] $p_{k_i} = \frac{phero_{k_i}^\alpha \cdot
		cost_{k_i}^\beta}{\sum\nolimits_{i=0}^n phero_{k_i}^\alpha \cdot
		cost_{k_i}^\beta}$ \pause
		\item [Pheromon-Update] $phero = phero +
		\left({\frac{1}{pathlength}}\right)^\gamma$  \pause
		\item [Pheromon-Evaporation] $phero = phero \cdot \left({ 1 -
\frac{phero_{k_i}}{phero_{max}}}\right)^\delta$
	\end{description}
}

\frame{
	\frametitle{Gewichtete Wahrscheinlichkeit}
	
	Beispiel: Kanten $\{K_1, K_2\}$; $phero_{k_1} = 5; phero_{k_2} = 10; \alpha =
2; \beta = 0$ 
	
	\[ p_{k_i} = \frac{phero_{k_i}^\alpha \cdot
			cost_{k_i}^\beta}{\sum\nolimits_{i=0}^n phero_{k_i}^\alpha \cdot
			cost_{k_i}^\beta} \]
	
	\only<1>{Alle Wahrscheinlichkeiten berechnen:}
	\only<2>{Gewichtung:}
	
	\begin{itemize}
		\only<1>{
			\item $p_{k_1} = \frac{5^2 (\cdot 1)}{5^2 + 10^2} = 0.2$
			\item $p_{k_2} = \frac{10^2 (\cdot 1)}{5^2 + 10^2} = 0.8$
		}
		\only<2>{
			\item $gew = (p_{k_1}, p_{k_1} + p_{k_2}) = (0.2, 1)$
			\item $r = random.random() = 0.46$
			\item $tuple_{index} = bisect(gew, r) = 1$
			\item $r$ öfters zwischen $0.2$ und $1$, manchmal zwischen $0$ und $0.2$
		}
	\end{itemize}
}

\frame{
	\frametitle{Einflussnahme}
	
	\only<1-2>{
		\begin{description}
		\item [$\alpha$] Verstärkung der Unterschiede zwischen Pheromonwerten \pause
		\item [$\beta$] Verstärkung der Unterschiede zwischen Kosten und Gewichtung
von Kosten
		\end{description}
	}
	
	\only<3-6>{
		\only<3-5>{
			\[ p_{k_i} = \frac{phero_{k_i}^\alpha \cdot
					cost_{k_i}^\beta}{\sum\nolimits_{i=0}^n phero_{k_i}^\alpha \cdot
					cost_{k_i}^\beta} \]
		}
	
		\only<3>{
			\begin{itemize}
			\item Alle Werte werden zwischen 0 und 1 normiert
			\item $phero_1 = 0.7; phero_2 = 0.3; cost_1 = 0.8; cost_2 = 0.5$
			\end{itemize}
		}
		
		\only<4>{
			\begin{itemize}
			\item Mit $\alpha = 1; \beta = 1$
			\end{itemize}
			
			\[ p_1 = \frac{0.7^1 \cdot 0.8^1}
				{0.7^1 \cdot 0.8^1 + 0.3^1 \cdot 0.5^1} = 0.79 \]
			\[ p_2 = \frac{0.3^1 \cdot 0.5^1}
				{0.7^1 \cdot 0.8^1 + 0.3^1 \cdot 0.5^1} = 0.21 \]
		}

		\only<5>{
			\begin{itemize}
			\item Mit $\alpha = 2; \beta = 2$
			\end{itemize}
			
			\[ p_1 = \frac{0.7^2 \cdot 0.8^2}
				{0.7^2 \cdot 0.8^2 + 0.3^2 \cdot 0.5^2} = 0.93 \]
			\[ p_2 = \frac{0.3^2 \cdot 0.5^2}
				{0.7^2 \cdot 0.8^2 + 0.3^2 \cdot 0.5^2} = 0.07 \]
		}
		
		\only<6>{
			\framesubtitle{Fazit}
			\begin{itemize}
			\item Die Unterschiede haben sich verstärkt
			\item Die Wahrscheinlichkeit ist grösser, dass die Ameise $p_1$ wählt
			\end{itemize}
		}
	}
}
