\documentclass{beamer}
\usepackage[T1]{fontenc}
\usepackage{textcomp}
\usepackage{ucs}
\usepackage[utf8x]{inputenc}
\usepackage[ngerman]{babel}
\usepackage{lmodern}
\usepackage{graphicx}
\usepackage{amsmath, amssymb, amsthm, array, ulem}
\usepackage{beamerthemesplit}
\usepackage{hyperref} % Zur Darstellung von URLs


\begin{document}
\title[Antz: Ameisenkolonie-Optimierung]{Antz: Ameisenkolonie-Optimierung} 
\subtitle{Software-Projekt 2}
\author{Bernhard Fuchs, Florian Müller} 
\date{4. Juni 2014} 

\frame{\titlepage} 

\frame[shrink]{
  \frametitle{Agenda}
  \vspace{10px}
  \tableofcontents
} 



\section{Einführung}
\frame{
  \frametitle{Überblick}
  \begin{itemize}
    \item Wahl des Themas: Ameisenalgorithmus zum Finden kürzester Pfade
    \item Wahl der Technologie: Python mit pyGame (Simulation mit graphischem  Interface)
    \item Fokus der Applikation
    \item Ameisenkolonie-Optimierung [Theorie]
    \begin{itemize}
	  \item ??? (folgt)
	  \item a
    \end{itemize}
   	\item a
    \begin{itemize}
	\item a
	\item a
    \end{itemize}
  \end{itemize}
}


\frame{
\frametitle{Hilfsmittel/Vorgehen}
  \begin{itemize}
    \item Tools erwähnen
    \item ??? (folgt)
  \end{itemize}
}

\section{Step By Step}

\frame{
	\frametitle{Step By Step Durchlauf}
	\begin{figure}
		\includegraphics<1>[width=0.8\textwidth]{img/antz-step-1.png}
		\includegraphics<2>[width=0.8\textwidth]{img/antz-step-2.png}
		\includegraphics<3>[width=0.8\textwidth]{img/antz-step-3.png}
		\includegraphics<4>[width=0.8\textwidth]{img/antz-step-4.png}
		\includegraphics<5>[width=0.8\textwidth]{img/antz-step-5.png}
		\includegraphics<6>[width=0.8\textwidth]{img/antz-step-6.png}
		\includegraphics<7>[width=0.8\textwidth]{img/antz-step-7.png}
		\includegraphics<8>[width=0.8\textwidth]{img/antz-step-8.png}
		\includegraphics<9>[width=0.8\textwidth]{img/antz-step-9.png}
	\end{figure}
}


\section{Applikation}
\frame{
\frametitle{Struktur}
???
}



\section{Fazit und Ausblick}
\frame{ 
\frametitle{Fazit und Ausblick}
\begin{large}\textbf{Erkenntnisse}\end{large}
\begin{itemize}
\item ACO-Algorithmus beim Finden des kürzesten Pfades oft recht langsam und mit suboptimaler Lösungsfindung (im Vergleich zu Dijkstra oder A*)
\item Rasche Anpassung an veränderte Gegebenheiten (Hindernisse)
\item Zeitintensiv: Anpassung an Parameter; Refactorings
\item Fokus gezielter vornehmen
\end{itemize}
\begin{large}\textbf{Ausblick}\end{large}
\begin{itemize}
\item Stärkere Angleichung an die Realität (z.B. mehrere Kolonien; Feinde; Überlebensstrategien)
\item Optimierung der Performance
\item Integration von Lösungen für weitere Probleme (z.B. TSP)
\end{itemize}}



\section{Bibliographie}
\frame{
\frametitle{Wichtige Literatur}
\begin{itemize}
\item ??? (folgt)
\end{itemize}
}



\section{Fragen}
\frame{
\frametitle{Fragen und Diskussion}
\begin{Large}Fragen? – Let's talk!\end{Large}
}


\end{document}
