% !TeX encoding=utf8
% !TeX spellcheck = en-US

\chapter{Projekt}

???

\section{Methodisches}


\subsection*{Planung:}

\begin{itemize}[noitemsep]
\item Drei Meilensteine geplant (erster à vier Wochen, zweite à je fünf Wochen)
\item Reserve von zwei Wochen am Ende eingeplant (insgesamt stehen 16 KW zur Verfügung)
\item Abgabe: 30.05.2014
\end{itemize}

\vspace*{1cm}

Vgl. unser Projektplanungstool \\

NB: Eventuell die Besprechungen mit Syrus integrieren (Feedback, Tipps, Stand, etc.) \\



\subsection{Technologien}

Programmiersprache Python, mit Framework pygame (weiter???)

\vspace*{1cm}


\subsection{Hilfsmittel}

Zur Projektverwaltung und Publikation wurde ein Repository auf GitHub (\url{https://github.com/jajadinimueter/antz}) verwendet. Dort legten wir auch die wesentlichen Dateien der Dokumentation ab (Ordner «Docs»). Für die Planung der Meilensteine und Iterationen wurde das Tool TargetProcess (http://jajadinimueter.tpondemand.com) verwendet.



\vspace*{1cm}

Zunächst eröffneten wir einen Account auf sharelatex.com, um gemeinsam an der Projektdokumentation mit LaTeX arbeiten zu können. Später entdeckten wir ein praktisches LaTeX-Template von Matthias Pospiech (\url{http://www.matthiaspospiech.de/latex/vorlagen/allgemein}), mit dem die Dokumenation geschrieben werden sollte. Da dafür die Möglichkeiten von sharelatex.com zu wenig ausgeprägt und zu langsam waren, wurde entschieden, die LaTeX-Dokumentation auch mit Git zu pflegen.

