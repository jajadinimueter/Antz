% !TeX encoding=utf8
% !TeX spellcheck = en-US

\chapter{Einführung, Allgemeines}

[Kapitelbezeichnungen in dieser Form umbenennen oder ganz weglassen] \\

[Einleitung ins Projekt, Allgemeines]

\vspace*{1cm}

\hspace*{3cm} Test Bibliographie (Zitatverweis): \cite[S. 115]{ds-ant}


\section{Anforderungen}

Vgl. die Präsentationsfolien und meine Notizen (offiziell und projektspezifisch) \\


\vspace*{1cm}


\section{Überblick}

Vorwort, Überlegungen, Schwierigkeiten \\

Zuerst wurde geprüft, ob das Projekt mit JavaScript und einer Game Framework wie quintus oder ??? als Website umgesetzt werden könnte (Canvas für 2D). Allerdings hat ein Test gezeigt, dass bereits ab einer Anzahl von 10'000 einzelnen Elementen (sprites) das Rendering in HTML5 bezüglich der Geschwindigkeit überhaupt nicht akzeptabel war. Daher wurde entschieden, den Algorithmus mit Python und dem Framework pygame umzusetzen, das auf der C-Library STL basiert und dessen Leistung sehr gut ist. Das fertige Programm wird so eine ausführbare Datei sein.

\vspace*{1cm}

Weiter unten: Systemspezifische Beschränkungen erwähnen (Leistung), evtl. auf Optimierungsparameter eingehen

\vspace*{1cm} 

\begin{itemize}[noitemsep]
\item PGU: Library für Slider und Textfelder, etc.
\item Optimierung: Anzahl Partikel; Speicherbedarf
\item Klassendiagramme und Mockups (von Hand) zeichnen
\item Zwei Modi (mehrere Render-Möglichkeiten des Algorithmus)
\item Anfangs: 2D; trotzdem 3D-Koordinaten einbauen (dafür: WebGL als weiterer Renderer)
\item Einbau von Hindernissen
\end{itemize}


Links, etc. dokumentieren


\vspace*{1cm}


\section{Theorie/Geschichtlicher Hintergrund} 

Evtl. kurz die Entwicklung der Algorithmen zur Ameisenkolonieoptimierung aufzeigen (vgl. die Literatur, Wikipedia-Artikel, u.ä.).


Die Ameisenkolonie-Optimierung (Ant Colony Optimization, ACO) richtet sich nach dem Verhalten von Ameisen bei der Wegfindung vom Nest zu Futterquellen, um damit verschiedene Probleme der kombinatorischen Optimierung zu lösen (vgl. \cite[S. 1]{sch-koa}). Dabei senden die Ameisen auf Futtersuche den Duftstoff Pheromon aus. Auf kürzeren Wegen ist die Pheromonkonzentration im Laufe der Zeit höher. Andere Ameisen richten sich bei ihrer Wegfindung nach der Höhe des Pheromons, so dass sie eher einen Weg mit hoher Pheromonkonzentration und damit einen kürzeren Pfad einschlagen. Es bildet sich eine sogenannte Ameisenstrasse aus.
Dieses Ameisenverhalten wurde auf Algorithmen übertragen. Der erste dieser Art wurde 1991 von Marco Dorigo publiziert, und zwar zur Lösung des Problems des Handlungsreisenden (Traveling Salesman Problem, TSP). 1999 steuerte Thomas Stützle entscheidende Ergänzungen bei (vgl. \cite{ds-ant}). \\\\

Beschreibung der Besonderheiten (Heuristik: nicht zwingend korrekte Lösung) \\

Mögliche Anwendungen beschreiben (vgl. auch Wikipedia) \\

Vgl. auch Schiller


\vspace*{1cm}
