% !TeX encoding=utf8
% !TeX spellcheck = en-US

\chapter{Einführung/Allgemeines}

[Einleitung ins Projekt, Allgemeines]


\subsection{Anforderungen}

???


\subsection{Theorie/Geschichtlicher Hintergrund} 

Evtl. kurz die Entwicklung der Algorithmen zur Ameisenkolonieoptimierung aufzeigen.

???


\vspace*{1cm}

Zuerst wurde geprüft, ob das Projekt mit JavaScript und einer Game Framework wie quintus oder ??? als Website umgesetzt werden könnte (Canvas für 2D). Allerdings hat ein Test gezeigt, dass bereits ab einer Anzahl von 10'000 einzelnen Elementen (sprites) das Rendering in HTML5 bezüglich der Geschwindigkeit überhaupt nicht akzeptabel war. Daher wurde entschieden, den Algorithmus mit Python und dem Framework pygame umzusetzen, das auf der C-Library STL basiert und dessen Leistung sehr gut ist. Das fertige Programm wird so eine ausführbare Datei sein.


\vspace*{1cm} 

\begin{itemize}[noitemsep]
\item PGU: Library für Slider und Textfelder, etc.
\item Optimierung: Anzahl Partikel; Speicherbedarf
\item Klassendiagramme und Mockups (von Hand) zeichnen
\item Zwei Modi (mehrere Render-Möglichkeiten des Algorithmus)
\item Anfangs: 2D; trotzdem 3D-Koordinaten einbauen (dafür: WebGL als weiterer Renderer)
\item Einbau von Hindernissen
\end{itemize}


Links, etc. dokumentieren


\vspace*{1cm}
