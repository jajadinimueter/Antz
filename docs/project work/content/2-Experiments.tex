% !TeX encoding=utf8
% !TeX spellcheck = en-US

\chapter{Applikation/Entwurf???}



Vorsichtsmassnahme, dass gewisse Ameisen sterben (weil z.B. 500 Ameisen, die sich auf einer Seite des Displays befinden, nicht zurück können, da sie den Weg zurück nicht nehmen können; sobald ein bester Pfad gefunden ist, sterben gewisse Ameisen nach einer gewissen Zeit, wenn sie nicht mehr mit Pheromon in Verbindung kommen) \\

\section{Mockups}

???

\vspace*{2cm}



\section{Klassendiagramme}



\section*{Ausbau}

\subsubsection*{Iteration 1}

Noch nicht implementiert, dass die Wegfindung abhängig von der Anzahl Ameisen, die durchlaufen, ist. \\

Evtl. noch eine Multi-Threaded-Lösung integrieren (sehr gut zu parallelisieren) \\

Ausserdem: schnellere Python-Interpreter wie pypy (oder für numerische Berechnungen); für z.B . 10'000 Nodes! \\

Idee: vielleicht noch Rauschen integrieren? Random von 20\,\% konvergiert überhaupt nicht mehr.

\vspace*{3cm}


Inputs von Syrus: \\

\begin{itemize}
\item Interessant wäre, das Maximum der Verlangsamung bei mehr Ameisen zu messen
\item Eventuell in Richtung Programming Tuning gehen? (Performance-Messungen: wie viel fällt auf welchen Bereich?) 
\item Für die Iterationen 2 und 3: Einsatz der Ressourcen planen (Andere Algorithmen; Optimierung; Vergleich mit Djikstra: Statistiken; vgl. Optimierungspotential bezüglich Zeit und Speicheraufwand, etc.); z.B. eher TSP (weil visuell für die Mitstudierenden ansprechender?) 
\end{itemize}