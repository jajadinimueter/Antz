%% !TeX encoding=utf8
%% !TeX spellcheck = de-DE


\chapter*{Abstract}

Die vorliegende Arbeit behandelt das Finden kürzester Pfade durch einen Graphen
unter Verwendung des Ameisenalgorithmus (Ant Colony Optimization). Dazu wurde
eine Simulation mit graphischem Interface in der Programmiersprache Python
geschrieben. Zur graphischen Darstellung wird das Game-Framework pyGame
verwendet. Das Programm unterstützt eine Anzahl von Konfigurationsmöglichkeiten,
um das Konvergenzverhalten des Algorithmus zu beeinflussen. Zudem können
einzelne Knoten zu Hindernissen umfunktioniert werden, wodurch die
Anpassungsfähigkeit des Algorithmus beobachtet werden kann. Die Konvergenz kann
in einem Chart nachverfolgt werden.

Fazit dieser Arbeit ist, dass der Ameisenalgorithmus beim Finden von Pfaden sehr
langsam ist und oft die ideale Lösung nicht trifft. Dennoch sind die Autoren der
Meinung, dass er eine Existenzberechtigung in dynamischen System hat, da er sich
relativ schnell an veränderte Verhältnisse anpassen kann, wie z.B. wenn ein
Knoten des Graphen verschoben oder blockiert wird. Ein direkter Vergleich mit
alternativen Algorithmen wie Dijkstra oder A* ist nicht Bestandteil dieser
Arbeit.