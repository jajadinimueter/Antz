\documentclass{article}

\usepackage{textcomp}
\usepackage{ucs}
\usepackage[utf8]{inputenc}
\usepackage[T1]{fontenc}
\usepackage[ngerman]{babel}
\usepackage{lmodern}
\usepackage{tabularx}
\usepackage{amsmath, amssymb, amsthm, array, ulem}

\setlength{\parindent}{0in} % kein Absatzeinzug in der ersten Zeile eines neuen Absatzes 

\usepackage{graphicx}
\usepackage{dsfont}
\usepackage{paralist} % für kompakte Listen
\usepackage{siunitx} % SI-Einheiten, Zahlendarstellung

\sisetup{exponent-product = \cdot} % Exponentendarstellung mit Punkt als Malzeichen

\usepackage{fancyhdr}
\pagestyle{fancy}
\lhead{Softwareprojekt 2}
\rhead{B. Fuchs, F. Müller}

\usepackage[headsep=1cm,left=2.5cm,right=3cm,bottom=3.5cm]{geometry}


\title{Antz}
\author{Bernhard Fuchs, Florian Müller}
\date{30. Mai 2014}

\begin{document}

\maketitle





\section{Einführung}

[Einleitung ins Projekt, Allgemeines]

\vspace*{1cm}

\begin{compactitem}
\item Technologie: JavaScript (mit Canvas für 2D), Framework quintus (für Games) oder pygame? (dann mit .exe-Datei); auch gezippt (mit HTML, Readme); auch Klassendiagramme
\item Zwei Modi (mehrere Render-Möglichkeiten des Algorithmus)
\item Anfangs: 2D; trotzdem 3D-Koordinaten einbauen (dafür: WebGL als weiterer Renderer)
\item Einbau von Hindernissen
\end{compactitem}


\vspace*{1cm}


\section{Methodisches}


\subsection*{Planung:}

\begin{compactitem}
\item Drei Meilensteine geplant (erster à vier Wochen, zweite à je fünf Wochen)
\item Reserve von zwei Wochen am Ende eingeplant (insgesamt stehen 16 KW zur Verfügung)
\item Abgabe: 30.05.2014
\end{compactitem}


\vspace*{1cm}


\subsection*{Tools/Hilfsmittel}

\begin{compactitem}
\item Repository auf Github: https://github.com/jajadinimueter/antz (Einbinden von Libraries, z.B. für die Vererbung in JavaScript)
\item [Dort einen Ordner «Docs» einrichten (für Latex-History; Bilder), etc.]
\item Planungstool: TargetProcess (http://jajadinimueter.tpondemand.com)
\item Für die gemeinsame Erarbeitung der Projektarbeit haben wir uns für http://sharelatex.com entschieden.
\end{compactitem}




\end{document}
